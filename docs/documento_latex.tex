\documentclass[12pt,a4paper]{article}

\usepackage[spanish]{babel}
\usepackage[utf8]{inputenc}
\usepackage{geometry}
\usepackage{graphicx}
\usepackage{listings}
\usepackage{xcolor}
\usepackage{hyperref}
\usepackage{fancyhdr}
\usepackage{setspace}

\geometry{margin=2.5cm}
\onehalfspacing

\lstset{
    language=Python,
    basicstyle=\ttfamily\small,
    keywordstyle=\color{blue},
    stringstyle=\color{red},
    commentstyle=\color{green!60!black},
    showstringspaces=false,
    breaklines=true
}

\begin{document}

% ------------------ PORTADA ------------------

\begin{titlepage}
    \centering
    \vspace*{2cm}

    \begin{figure}
        \centering
        \includegraphics[width=0.5\linewidth]{pngwing.com (9).png}

        
    \end{figure}
    
    {\Large \textbf{UNIVERSIDAD DISTRITAL FJDC}}\\[0.5cm]
    {\large Facultad de Ingeniería}\\[2cm]
    
    \rule{\linewidth}{0.5mm} \\[0.4cm]
    {\Huge \textbf{Biblioteca datetime en Python}}\\[0.3cm]
    \rule{\linewidth}{0.5mm} \\[2cm]
    
    \textbf{Asignatura:} Probabilidad y Estadística \\[0.5cm]
    \textbf{Docente:} Alberto Acosta Lopez \\[0.5cm]
    \textbf{Estudiante:} Daniel Alejandro Castro Becerra\\[0.5cm]
    \textbf{Código:} 20242020271 \\[2cm]
    
    \vfill
    
\end{titlepage}

% ------------------ INDICE ------------------

\tableofcontents
\newpage

% ------------------ INTRODUCCION ------------------

\section{Introducción}

La biblioteca \texttt{datetime} es un módulo estándar de Python que permite manipular fechas y horas de manera eficiente. 
Es una herramienta fundamental en el desarrollo de software moderno, ya que prácticamente todos los sistemas informáticos dependen del manejo correcto del tiempo.

Desde sistemas bancarios hasta plataformas educativas, el control preciso de fechas y horas garantiza la integridad de los datos y la correcta ejecución de procesos automáticos.

% ------------------ COMPONENTES ------------------

\section{Componentes Principales}

\subsection{Clase date}

Representa una fecha específica compuesta por año, mes y día. 
Se utiliza cuando no es necesario trabajar con información de hora.

Ejemplo:

\begin{lstlisting}
from datetime import date

fecha = date(2026, 5, 20)
print(fecha)
\end{lstlisting}

\subsection{Clase time}

Representa únicamente la hora (hora, minutos, segundos y microsegundos).

\begin{lstlisting}
from datetime import time

hora = time(14, 30, 0)
print(hora)
\end{lstlisting}

\subsection{Clase datetime}

Combina fecha y hora en un solo objeto. Es la clase más utilizada.

\begin{lstlisting}
from datetime import datetime

ahora = datetime.now()
print(ahora)
\end{lstlisting}

\subsection{Clase timedelta}

Permite realizar operaciones matemáticas con fechas y horas.

\begin{lstlisting}
from datetime import datetime, timedelta

hoy = datetime.now()
manana = hoy + timedelta(days=1)

print("Hoy:", hoy)
print("Mañana:", manana)
\end{lstlisting}

% ------------------ FUNCIONES IMPORTANTES ------------------

\section{Funciones Principales}

\subsection{datetime.now()}

Devuelve la fecha y hora actual del sistema.

\subsection{strftime()}

Convierte un objeto datetime en una cadena de texto con formato específico.

\begin{lstlisting}
from datetime import datetime

ahora = datetime.now()
formato = ahora.strftime("%d/%m/%Y %H:%M")
print(formato)
\end{lstlisting}

\subsection{strptime()}

Convierte una cadena de texto en un objeto datetime.

\begin{lstlisting}
from datetime import datetime

texto = "25/12/2026"
fecha = datetime.strptime(texto, "%d/%m/%Y")
print(fecha)
\end{lstlisting}

% ------------------ EJEMPLO PRACTICO COMPLETO ------------------

\section{Ejemplo Práctico Completo}

Supongamos que queremos calcular cuántos días faltan para finalizar el año.

\begin{lstlisting}
from datetime import datetime

hoy = datetime.now()
fin_ano = datetime(hoy.year, 12, 31)

diferencia = fin_ano - hoy

print("Hoy es:", hoy.strftime("%d/%m/%Y"))
print("Faltan", diferencia.days, "días para finalizar el año.")
\end{lstlisting}

Este tipo de cálculo es común en aplicaciones financieras, sistemas de planificación y gestión empresarial.

% ------------------ IMPORTANCIA ------------------

\section{Importancia en Sistemas Reales}

El manejo adecuado del tiempo es esencial en:

\begin{itemize}
    \item Sistemas bancarios para registrar pagos e intereses.
    \item Plataformas educativas para fechas de entrega.
    \item Sistemas empresariales para auditoría y control.
    \item Automatización de procesos programados.
\end{itemize}

Un error en el manejo de fechas puede generar pérdidas económicas, inconsistencias de datos o fallos en procesos críticos.

% ------------------ PROBLEMAS COMUNES ------------------

\section{Problemas Comunes}

\begin{itemize}
    \item Confusión entre formatos de fecha (DD/MM/AAAA vs MM/DD/AAAA).
    \item Diferencias de zonas horarias.
    \item Manejo incorrecto de años bisiestos.
    \item Falta de validación en datos ingresados por el usuario.
\end{itemize}

% ------------------ BUENAS PRACTICAS ------------------

\section{Buenas Prácticas}

\begin{itemize}
    \item Definir un formato estándar para fechas.
    \item Validar datos de entrada.
    \item Considerar el uso de UTC en sistemas distribuidos.
    \item Documentar el formato utilizado en bases de datos.
\end{itemize}

% ------------------ CONCLUSIONES ------------------

\section{Conclusiones}

La biblioteca datetime es una herramienta esencial en el desarrollo de software. 
Permite gestionar fechas y horas de manera precisa, facilitando la creación de sistemas confiables y automatizados.

Su correcta implementación reduce errores, mejora la integridad de los datos y garantiza el funcionamiento adecuado de aplicaciones críticas.

% ------------------ BIBLIOGRAFIA ------------------

\section{Bibliografía}

\begin{itemize}

\item Python Software Foundation. (s. f.). 
\textit{datetime - Basic date and time types}. 
Python Documentation. 
Disponible en: https://docs.python.org/3/library/datetime.html

\item Aprende con Alf. (s. f.). 
\textit{La libreria datetime}. 
Disponible en: https://aprendeconalf.es/docencia/python/manual/datetime/

\item Ortiz Ordonez, J. (2019, septiembre 13). 
\textit{Python - Ejercicio 3: Obtener la fecha y hora actuales del sistema con el modulo datetime} [Video]. 
YouTube. 
Disponible en: https://www.youtube.com/watch?v=WxQrV_aIXaE

\end{itemize}

\end{document}
